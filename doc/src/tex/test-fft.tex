\subsection{TestFFT}
\index{Applications!TestFFT}
\index{Testing Applications!TestFFT}

$Revision: 1.5 $

\api{TestFFT} is one of the testing applications in {\marf}.
It aims to test how the FFT (Fast Fourier Transform) algorithm works in
\marf{FeatureExtraction} by loading simulated or real wave sound samples.

From user's point of view, \api{TestFFT} provides with usage and loaders
command-line options. By entering \texttt{--help}, or \texttt{-h}, or
even no arguments, the usage information will be displayed. It explains
the arguments used in \api{TestFFT}. Another argument is the type of
loader. \api{TestFFT} uses two types of loaders, \api{Sineloader} and \api{WAVLoader}.
Users can enter \texttt{--sine} or \texttt{--wave} as the second argument.
The argument \texttt{--sine} uses \api{SineLoader} that will generate a
plain sine wave to be fed to a generic preprocessor.
If the argument \texttt{--wave} is specified, the application uses \api{WAVLoader}
to load a wave sample from file. In the latter case, users need to supply
one more argument -- sample file in the format of *.wav to be loaded by \api{WAVLoader}.
After selecting all necessary arguments, user can run and get the output of \api{TestFFT} within seconds.

\noindent
Complete usage information:

\vspace{15pt}
\hrule
\begin{verbatim}

Usage:
    java TestFFT --help | -h
        displays this help and exits

    java TestFFT --version
        displays application and MARF versions

    java TestFFT --sine
        runs the FFT algorithm on a generated sine wave

    java TestFFT --wave <wave-file>
        runs the FFT algorithm against specified voice sample

\end{verbatim}

\hrule
\vspace{15pt}


The application is made to exercise the {\marf}'s
FFT-based feature extraction algorithm located in the \api{marf.FeatureExtraction.FFT} package.
Additionally, the following {\marf} modules are utilized:

\begin{enumerate}
\item
\api{marf.MARF}

\item
\api{marf.util.OptionProcessor}

\item
\api{marf.Storage.Sample}

\item
\api{marf.Storage.SampleLoader}

\item
\api{marf.Storage.WAVLoader}

\item
\api{marf.Preprocessing.Dummy}
\end{enumerate}

The main \api{MARF} module enumerates incoming sample file format as
\api{WAV}, \api{SINE}. \api{OptionProcessor} helps maintain and validate
command-line options. \api{Sample} maintains and processed incoming
sample data. \api{SampleLoader} provides sample loading interface for
all the {\marf} loaders. It must be overridden by a concrete sample loader
such as \api{SineLoader} or \api{WAVLoader}. \api{Preprocessing} does general
preprocessing such with \api{preprocess()} (overridden by \api{Dummy}), \api{normalize()},
\api{removeNoise()} and \api{removeSilence()} out of which for this application the
former two are used. Then, processed data will be serve for an parameter of \api{FeatureExtraction.FFT.FFT}
to extract sample's features. In the end, above modules work together to test the work of the FFT
algorithm and produce the output to STDOUT.

As we know, the output of \api{TestFFT} extracts the features data by FFT feature extraction algorithm.
It gives both users and programmers direct information of the effect of MARF feature extraction,
and can be compared with the expected output in the \file{expected} folder to detect any errors
if the underlying algorithm has been changed.

% EOF
